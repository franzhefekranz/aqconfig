%% Generated by Sphinx.
\def\sphinxdocclass{report}
\documentclass[a4paper,10pt,english]{sphinxmanual}
\ifdefined\pdfpxdimen
   \let\sphinxpxdimen\pdfpxdimen\else\newdimen\sphinxpxdimen
\fi \sphinxpxdimen=.75bp\relax

\PassOptionsToPackage{warn}{textcomp}
\usepackage[utf8]{inputenc}
\ifdefined\DeclareUnicodeCharacter
% support both utf8 and utf8x syntaxes
  \ifdefined\DeclareUnicodeCharacterAsOptional
    \def\sphinxDUC#1{\DeclareUnicodeCharacter{"#1}}
  \else
    \let\sphinxDUC\DeclareUnicodeCharacter
  \fi
  \sphinxDUC{00A0}{\nobreakspace}
  \sphinxDUC{2500}{\sphinxunichar{2500}}
  \sphinxDUC{2502}{\sphinxunichar{2502}}
  \sphinxDUC{2514}{\sphinxunichar{2514}}
  \sphinxDUC{251C}{\sphinxunichar{251C}}
  \sphinxDUC{2572}{\textbackslash}
\fi
\usepackage{cmap}
\usepackage[T1]{fontenc}
\usepackage{amsmath,amssymb,amstext}
\usepackage{babel}



\usepackage{times}
\expandafter\ifx\csname T@LGR\endcsname\relax
\else
% LGR was declared as font encoding
  \substitutefont{LGR}{\rmdefault}{cmr}
  \substitutefont{LGR}{\sfdefault}{cmss}
  \substitutefont{LGR}{\ttdefault}{cmtt}
\fi
\expandafter\ifx\csname T@X2\endcsname\relax
  \expandafter\ifx\csname T@T2A\endcsname\relax
  \else
  % T2A was declared as font encoding
    \substitutefont{T2A}{\rmdefault}{cmr}
    \substitutefont{T2A}{\sfdefault}{cmss}
    \substitutefont{T2A}{\ttdefault}{cmtt}
  \fi
\else
% X2 was declared as font encoding
  \substitutefont{X2}{\rmdefault}{cmr}
  \substitutefont{X2}{\sfdefault}{cmss}
  \substitutefont{X2}{\ttdefault}{cmtt}
\fi


\usepackage[Bjarne]{fncychap}
\usepackage{sphinx}

\fvset{fontsize=\small}
\usepackage{geometry}

% Include hyperref last.
\usepackage{hyperref}
% Fix anchor placement for figures with captions.
\usepackage{hypcap}% it must be loaded after hyperref.
% Set up styles of URL: it should be placed after hyperref.
\urlstyle{same}

\usepackage{sphinxmessages}




\title{Aqconfig Documentation}
\date{Nov 06, 2020}
\release{1.0}
\author{Michael Taxis}
\newcommand{\sphinxlogo}{\vbox{}}
\renewcommand{\releasename}{Release}
\makeindex
\begin{document}

\ifdefined\shorthandoff
  \ifnum\catcode`\=\string=\active\shorthandoff{=}\fi
  \ifnum\catcode`\"=\active\shorthandoff{"}\fi
\fi

\pagestyle{empty}
\sphinxmaketitle
\pagestyle{plain}
\sphinxtableofcontents
\pagestyle{normal}
\phantomsection\label{\detokenize{index::doc}}


The program edits the configuration for an acquisition tool for Siemens Simatic S7 PLCs.
It was written in Python with Tkinter as GUI library.
To build the GUI a GUI Builder named PAGE was used.
\begin{quote}\begin{description}
\item[{Homepage}] \leavevmode
\sphinxurl{http://www.taxis-instruments.de}

\item[{Contact}] \leavevmode
\textless{}aqserver at taxis-instruments dot de\textgreater{}

\item[{Author}] \leavevmode
Michael Taxis \textless{}michael at taxis-instruments dot de\textgreater{}

\end{description}\end{quote}

Copyright of this document:
\begin{quote}\begin{description}
\item[{Copyright}] \leavevmode
\sphinxhref{http://creativecommons.org/licenses/by-sa/3.0/}{CC BY-SA 3.0}

\end{description}\end{quote}

Copyright of the program:
\begin{quote}\begin{description}
\item[{Copyright}] \leavevmode
\sphinxhref{https://www.gnu.org/licenses/lgpl-3.0-standalone.html}{LESSER GNU GENERAL PUBLIC LICENSE 3.0}

\end{description}\end{quote}

Please note that soft- and hardware designations and brand names of their respective companies  used in this document are  generally subject to trademark protection or patent protection.


\chapter{About}
\label{\detokenize{about:about}}\label{\detokenize{about::doc}}
With this program you can edit configuration files for the data aqcuisition program aqserver.
Its GUI is written in \sphinxhref{http://www.python.org/}{Python} and \sphinxhref{https://docs.python.org/3/library/tkinter.html}{Tkinter}.
To build the GUI a GUI Builder named \sphinxhref{http://page.sourceforge.net/}{PAGE} was used.

This was my first (finished) Python (3) application with a GUI. I always use my
(private) projects to learn something new.
Lots of web pages and lots of \sphinxhref{https://stackoverflow.com/}{Stack overflow} entries
have been visited to find solutions to my problems.
Thanks a lot to the community of coders and developpers.


\chapter{Main interface}
\label{\detokenize{main:main-interface}}\label{\detokenize{main::doc}}

\section{Menu and toolbar}
\label{\detokenize{main:menu-and-toolbar}}
\noindent{\hspace*{\fill}\sphinxincludegraphics{{toolbar}.png}\hspace*{\fill}}

Menu and toolbar show the same functions. left part of the toolbar shows the
File menu functions, right part shows the Help menu functions.
\begin{enumerate}
\sphinxsetlistlabels{\arabic}{enumi}{enumii}{}{.}%
\item {} \begin{description}
\item[{File}] \leavevmode\begin{enumerate}
\sphinxsetlistlabels{\arabic}{enumii}{enumiii}{}{.}%
\item {} 
New
\begin{quote}

\noindent{\sphinxincludegraphics[width=32\sphinxpxdimen,height=32\sphinxpxdimen]{{btn-document-new}.png}\hspace*{\fill}}

Clears all fields for a fresh config file
\end{quote}

\item {} 
Open …
\begin{quote}

\noindent{\sphinxincludegraphics[width=32\sphinxpxdimen,height=32\sphinxpxdimen]{{btn-document-open}.png}\hspace*{\fill}}

Opens an existing configuration file
\end{quote}

\item {} 
Save
\begin{quote}

\noindent{\sphinxincludegraphics[width=32\sphinxpxdimen,height=32\sphinxpxdimen]{{btn-document-save}.png}\hspace*{\fill}}

Save the open file
\end{quote}

\item {} 
Save as …
\begin{quote}

\noindent{\sphinxincludegraphics[width=32\sphinxpxdimen,height=32\sphinxpxdimen]{{btn-document-save-as}.png}\hspace*{\fill}}

Lets you save the open file under a new name
\end{quote}

\item {} 
Settings
\begin{quote}

\noindent{\sphinxincludegraphics[width=32\sphinxpxdimen,height=32\sphinxpxdimen]{{btn-configure}.png}\hspace*{\fill}}

Opens a new dialog window, where you can adjust the program settings
\end{quote}

\item {} 
Save *.bat
\begin{quote}

\noindent{\sphinxincludegraphics[width=32\sphinxpxdimen,height=32\sphinxpxdimen]{{btn-ms_dos_batch_file}.png}\hspace*{\fill}}

Saves a batch file next to the config file, that you can use to start
Aqserver directly with this config-file
\end{quote}

\item {} 
Quit
\begin{quote}

\noindent{\sphinxincludegraphics[width=32\sphinxpxdimen,height=32\sphinxpxdimen]{{btn-close}.png}\hspace*{\fill}}

Closes the configuration program
\end{quote}

\end{enumerate}

\end{description}

\item {} \begin{description}
\item[{Help}] \leavevmode\begin{enumerate}
\sphinxsetlistlabels{\arabic}{enumii}{enumiii}{}{.}%
\setcounter{enumii}{7}
\item {} 
Help contents
\begin{quote}

\noindent{\sphinxincludegraphics[width=32\sphinxpxdimen,height=32\sphinxpxdimen]{{btn-help-contents}.png}\hspace*{\fill}}

Opens the help file i a browser
\end{quote}

\item {} 
Context help
\begin{quote}

\noindent{\sphinxincludegraphics[width=32\sphinxpxdimen,height=32\sphinxpxdimen]{{btn-help-contextual}.png}\hspace*{\fill}}

switches the mouse pointer to a arrow with question mark. Click on a element
in the interface and help for this element will be displayed in the browser
\end{quote}

\item {} 
About
\begin{quote}

\noindent{\sphinxincludegraphics[width=32\sphinxpxdimen,height=32\sphinxpxdimen]{{btn-help-about}.png}\hspace*{\fill}}

Shows an About box for the program
\end{quote}

\end{enumerate}

\end{description}

\end{enumerate}

\noindent{\sphinxincludegraphics{{about}.png}\hspace*{\fill}}


\section{Statusbar}
\label{\detokenize{main:statusbar}}
\noindent{\hspace*{\fill}\sphinxincludegraphics{{statusbar}.png}\hspace*{\fill}}

The statusbar shows the name of the open config file and whether it has been modified


\chapter{Information tab}
\label{\detokenize{info:information-tab}}\label{\detokenize{info::doc}}
\noindent{\hspace*{\fill}\sphinxincludegraphics{{01_info}.png}\hspace*{\fill}}

In this tab we can enter some general information about the data recording, to identify the situation later on
\begin{enumerate}
\sphinxsetlistlabels{\arabic}{enumi}{enumii}{}{.}%
\item {} \begin{description}
\item[{Customer:}] \leavevmode
here we enter the customers name or some other information

\end{description}

\item {} \begin{description}
\item[{Creator:}] \leavevmode
enter your name

\end{description}

\item {} \begin{description}
\item[{Machine:}] \leavevmode
information about the machine (e.g. type, year, machine number)

\end{description}

\item {} \begin{description}
\item[{Order:}] \leavevmode
enter your order number

\end{description}

\item {} \begin{description}
\item[{Remarks:}] \leavevmode
enter a description for the recording, e.g. what was the problem, how did you solve it, result

\end{description}

\end{enumerate}


\chapter{Communication tab}
\label{\detokenize{com:communication-tab}}\label{\detokenize{com::doc}}
\noindent{\hspace*{\fill}\sphinxincludegraphics{{02_com}.png}\hspace*{\fill}}

Here we enter the communication details, how to reach the PLC
\begin{enumerate}
\sphinxsetlistlabels{\arabic}{enumi}{enumii}{}{.}%
\item {} \begin{description}
\item[{Demo:}] \leavevmode
if checked, we run the program in demo mode, no real communication, random values will be created

\end{description}

\item {} \begin{description}
\item[{IP address:}] \leavevmode
enter the IP address of the PLC

\end{description}

\item {} \begin{description}
\item[{Rack no.:}] \leavevmode
enter the rack number of the PLC

\end{description}

\item {} \begin{description}
\item[{Slot no.:}] \leavevmode
enter the slot number for the CPU in the rack, can be found in HW Config of Step7

\end{description}

\end{enumerate}

\noindent{\hspace*{\fill}\sphinxincludegraphics{{hwconfig}.png}\hspace*{\fill}}


\chapter{Miscellaneous tab}
\label{\detokenize{misc:miscellaneous-tab}}\label{\detokenize{misc::doc}}
\noindent{\sphinxincludegraphics{{03_misc}.png}\hspace*{\fill}}
\begin{enumerate}
\sphinxsetlistlabels{\arabic}{enumi}{enumii}{}{.}%
\item {} \begin{description}
\item[{datafile prefix}] \leavevmode
here you enter the name of the datafile, archives will additionally
have a timestamp in the name.

\end{description}

\item {} \begin{description}
\item[{storage path}] \leavevmode
here you can enter the path where you want the archives to be stored.
With the button a dialog willl open where you can pick a directory
\begin{quote}

\noindent{\sphinxincludegraphics{{dialog-storage-path}.png}\hspace*{\fill}}
\end{quote}

With this dialog you can select the storage path

\end{description}

\item {} \begin{description}
\item[{use….}] \leavevmode
With this checkbox you can select whether the archives should be stored under a directory structure by day

\end{description}

\item {} \begin{description}
\item[{scantime}] \leavevmode
enter the scan interval in milliseconds

\end{description}

\end{enumerate}


\chapter{Trigger tab}
\label{\detokenize{trigger:trigger-tab}}\label{\detokenize{trigger::doc}}

\section{Trigger (Wikipedia):}
\label{\detokenize{trigger:trigger-wikipedia}}
In physics, a trigger is a system that controls the data collection of an experiment.
Often it is not possible or not important to continuously record all measured values
in experiments. In this case only a part of the measured values is read out and based
on this, it is decided when the remaining data should be read out.

In our case we use the trigger to start a new datafile.

\noindent{\sphinxincludegraphics{{04_trigger}.png}\hspace*{\fill}}
\begin{enumerate}
\sphinxsetlistlabels{\arabic}{enumi}{enumii}{}{.}%
\item {} \begin{description}
\item[{trigger checkbox:}] \leavevmode
switch trigger function on or off

\end{description}

\item {} \begin{description}
\item[{trigger signal:}] \leavevmode
choose the trigger from available signals with the combobox

\sphinxincludegraphics{{04_trigger-signal}.png}

combobox with trigger signals

\end{description}

\item {} \begin{description}
\item[{trigger condition:}] \leavevmode
choose the comparison operator for the trigger

\end{description}

\item {} \begin{description}
\item[{trigger value:}] \leavevmode
value the trigger signal is compared with

\end{description}

\item {} \begin{description}
\item[{pre-trigger:}] \leavevmode
time before the trigger, where signals will go into new file

\end{description}

\item {} \begin{description}
\item[{post-trigger:}] \leavevmode
time after trigger, where signals will go into old file

\end{description}

\end{enumerate}

\begin{sphinxadmonition}{note}{Note:}
Pre- and post-trigger are used to get an overlap between old and new datafile
\end{sphinxadmonition}


\chapter{Debug tab}
\label{\detokenize{debug:debug-tab}}\label{\detokenize{debug::doc}}
\noindent{\sphinxincludegraphics{{05_debug}.png}\hspace*{\fill}}
\begin{enumerate}
\sphinxsetlistlabels{\arabic}{enumi}{enumii}{}{.}%
\item {} \begin{description}
\item[{Level}] \leavevmode
select the debugging level (for aqserver), where “1 - INFO” is the most verbose

\end{description}

\item {} \begin{description}
\item[{Files to keep}] \leavevmode
Give a number of logging files that will be kept.

\end{description}

\end{enumerate}


\chapter{Values tab}
\label{\detokenize{values:values-tab}}\label{\detokenize{values::doc}}

\section{Values}
\label{\detokenize{values:values}}
\noindent{\sphinxincludegraphics{{06_value}.png}\hspace*{\fill}}
\begin{enumerate}
\sphinxsetlistlabels{\arabic}{enumi}{enumii}{}{.}%
\item {} \begin{description}
\item[{Area of values}] \leavevmode
select the area of the value from the following
\begin{enumerate}
\sphinxsetlistlabels{\alph}{enumii}{enumiii}{}{.}%
\item {} \begin{description}
\item[{Data}] \leavevmode
value is from the data area, you have to select also a datablock number and the address

\end{description}

\item {} \begin{description}
\item[{Output}] \leavevmode
value is from the output area, you have to select also  the address.
Note that datablock number is greyed out.

\end{description}

\item {} \begin{description}
\item[{Input}] \leavevmode
value is from the input area, you have to select also  the address.
Note that datablock number is greyed out.

\end{description}

\item {} \begin{description}
\item[{Flag}] \leavevmode
value is from the flag area, you have to select also  the address.
Note that datablock number is greyed out.

\end{description}

\item {} \begin{description}
\item[{Timer}] \leavevmode
value is from the timer area, you have to select also  the address.
Note that datablock number is greyed out.
Also the only allowed format is Int 16 bit

\end{description}

\item {} \begin{description}
\item[{Counter}] \leavevmode
value is from the counter area, you have to select also  the address.
Note that datablock number is greyed out.
Also the only allowed format is Int 16 bit

\end{description}

\end{enumerate}

\end{description}

\item {} \begin{description}
\item[{Data block number}] \leavevmode
enter / select the number of the datablock

\end{description}

\item {} \begin{description}
\item[{Address}] \leavevmode
enter the address of the value

\end{description}

\item {} \begin{description}
\item[{Format}] \leavevmode
select the format of the value with the radio buttons
\begin{enumerate}
\sphinxsetlistlabels{\alph}{enumii}{enumiii}{}{.}%
\item {} \begin{description}
\item[{Real 32 bit}] \leavevmode
value is a real (float) number , with a length of 32 bit (REAL)
Note that bottom notebook switches automatically to the value tab.

\end{description}

\item {} \begin{description}
\item[{Dint 32 bit}] \leavevmode
value is a integer number, with a length of 32 bit (DINT)
Note that bottom notebook switches automatically to the value tab.

\end{description}

\item {} \begin{description}
\item[{Int 16 bit}] \leavevmode
value is a integer number, with a length of 16 bit (WORD)
Note that bottom notebook switches automatically to the value tab.

\end{description}

\item {} \begin{description}
\item[{Int 8 bit}] \leavevmode
value is a integer number, with a length of 8 bit (BYTE)
Note that bottom notebook switches automatically to the value tab.

\end{description}

\item {} \begin{description}
\item[{Bool}] \leavevmode
value is a boolean value (TRUE/FALSE). Note that bottom notebook
switches automatically to the bool tab.

\end{description}

\end{enumerate}

\end{description}

\end{enumerate}


\subsection{Value}
\label{\detokenize{values:value}}
\noindent{\sphinxincludegraphics{{06_values-value}.png}\hspace*{\fill}}
\begin{enumerate}
\sphinxsetlistlabels{\arabic}{enumi}{enumii}{}{.}%
\item {} \begin{description}
\item[{Name}] \leavevmode
The name for the value

\begin{sphinxadmonition}{note}{Note:}
Names have to be unique. No duplicate entries are allowed in the list!
\end{sphinxadmonition}

\end{description}

\item {} \begin{description}
\item[{Gain}] \leavevmode
The value coming from the PLC is multiplied with this value
This can be used if several signals have to be compared
but have different range (but maybe the same shape). Leave 1.0 for no effect (default setting).

\end{description}

\item {} \begin{description}
\item[{Offset}] \leavevmode
This value is added to the result of (value coming from the PLC multiplied by gain)
Use this to shift the value up or down. Leave 0.0 for no effect (default setting)

\end{description}

\item {} \begin{description}
\item[{Unit}] \leavevmode
Enter a unit of the measured value, e.g. m\(\sp{\text{3}}\), bar, psi…

\end{description}

\end{enumerate}


\subsection{Bool}
\label{\detokenize{values:bool}}
Smallest value that can be read from the PLC is a Byte.
So in order to get a boolean value we have to get a byte and then separate it to 8 booleans.

\noindent{\sphinxincludegraphics{{07_values-bool}.png}\hspace*{\fill}}
\begin{enumerate}
\sphinxsetlistlabels{\arabic}{enumi}{enumii}{}{.}%
\item {} \begin{description}
\item[{name for every bit}] \leavevmode
The name for the bits (also unique)

\end{description}

\item {} \begin{description}
\item[{Gain for every bit}] \leavevmode
Gain does not make much sense, but we have it anyway. Leave 1 for no effect (default setting)

\end{description}

\item {} \begin{description}
\item[{Offset for every bit}] \leavevmode
This value is added to the result of (value coming from the PLC multiplied by gain).
Can be useful if several boolean values are in one graph, to separate them.
Leave 0 for no effect (default setting)

\end{description}

\end{enumerate}

\begin{sphinxadmonition}{note}{Note:}
There is no unit for a boolean value
\end{sphinxadmonition}


\subsection{Buttons}
\label{\detokenize{values:buttons}}\begin{enumerate}
\sphinxsetlistlabels{\arabic}{enumi}{enumii}{}{.}%
\item {} \begin{description}
\item[{Clear}] \leavevmode
This button clears the entry fields and sets some defaults

\end{description}

\item {} \begin{description}
\item[{Edit}] \leavevmode
This button switches to the list tab, sop you select a value/row for edit

\end{description}

\item {} \begin{description}
\item[{Save}] \leavevmode
Changes to a selected value will be saved to the list.
If no value was selected previously , then button is greyed out

\end{description}

\item {} \begin{description}
\item[{Add}] \leavevmode
Values entered to the fields will be added to the end of the list

\end{description}

\end{enumerate}


\subsubsection{Duplicate names}
\label{\detokenize{values:duplicate-names}}
All value names must be unique. If a name entered to the Name field
or to the Name0..7 field for the boolean is already in the list, a
dialog will open and show the duplicates. Enter a different name for the shown ones
and save/add again.

\noindent{\sphinxincludegraphics{{dialog-duplicate}.png}\hspace*{\fill}}


\chapter{List tab}
\label{\detokenize{list:list-tab}}\label{\detokenize{list::doc}}
\noindent{\sphinxincludegraphics{{08_list}.png}\hspace*{\fill}}
\begin{enumerate}
\sphinxsetlistlabels{\arabic}{enumi}{enumii}{}{.}%
\item {} \begin{description}
\item[{Edit}] \leavevmode
Edit the selected value/row from the list. Will open this value in the values tab

\end{description}

\item {} \begin{description}
\item[{Delete}] \leavevmode
Deletes the selected value from the list. Has to be confirmed.

\end{description}

\end{enumerate}

\begin{sphinxadmonition}{note}{Note:}
Deleting a value from the list can destroy the display of a datafile in KST2,
because it is based on the order of the values in the datafiel. When we change the
order by deleting a value from the list, then display will not work anymore.
\end{sphinxadmonition}


\chapter{Settings}
\label{\detokenize{settings:settings}}\label{\detokenize{settings::doc}}
\noindent{\sphinxincludegraphics{{dialog-settings}.png}\hspace*{\fill}}
\begin{enumerate}
\sphinxsetlistlabels{\arabic}{enumi}{enumii}{}{.}%
\item {} \begin{description}
\item[{Button Path to Aqserver.exe}] \leavevmode
Here we have to select the directory where the Aqserver.exe is located.
\begin{quote}

\noindent{\sphinxincludegraphics{{dialog-path}.png}\hspace*{\fill}}
\end{quote}

Default location is “C:Program Files (x86)Aqserver”

\end{description}

\item {} \begin{description}
\item[{Label Path to Aqserver.exe}] \leavevmode
The label shows the directory

\end{description}

\item {} \begin{description}
\item[{Language combobox}] \leavevmode
Select language for the program from the combobox

\end{description}

\item {} \begin{description}
\item[{Ok}] \leavevmode
Click to accept settings

\end{description}

\item {} \begin{description}
\item[{Cancel}] \leavevmode
Click to close the dialog and leave previous settings

\end{description}

\end{enumerate}


\chapter{Changelog}
\label{\detokenize{changelog:changelog}}\label{\detokenize{changelog::doc}}\begin{itemize}
\item {} 
05 October 2020
\begin{itemize}
\item {} 
0.1.0alpha
\begin{quote}

first version
\end{quote}

\end{itemize}

\end{itemize}


\chapter{Licensing}
\label{\detokenize{license:licensing}}\label{\detokenize{license::doc}}
Aqserver is distributed as python source code or Windows setup program under
\sphinxhref{https://www.gnu.org/licenses/lgpl-3.0-standalone.html}{Lesser General Public License version 3.0 (LGPLv3)}

Basically this means that you can distribute your commercial software
linked with Aqconfig without the requirement to distribute the source
code of your application and without the requirement that your
application be itself distributed under LGPL.

A small mention to the project or the author is however appreciated
if you include it in your applications.


\section{Disclaimer of Warranty}
\label{\detokenize{license:disclaimer-of-warranty}}
THERE IS NO WARRANTY FOR THE PROGRAM, TO THE EXTENT PERMITTED BY
APPLICABLE LAW. EXCEPT WHEN OTHERWISE STATED IN WRITING THE
COPYRIGHT HOLDERS AND/OR OTHER PARTIES PROVIDE THE PROGRAM “AS IS”
WITHOUT WARRANTY OF ANY KIND, EITHER EXPRESSED OR IMPLIED,
INCLUDING, BUT NOT LIMITED TO, THE IMPLIED WARRANTIES OF
MERCHANTABILITY AND FITNESS FOR A PARTICULAR PURPOSE. THE ENTIRE
RISK AS TO THE QUALITY AND PERFORMANCE OF THE PROGRAM IS WITH YOU.
SHOULD THE PROGRAM PROVE DEFECTIVE, YOU ASSUME THE COST OF ALL
NECESSARY SERVICING, REPAIR OR CORRECTION.

IF ANYONE BELIEVES THAT, WITH THIS WEBSITE OR WITH AQCONFIG PROJECT
SOME COPYRIGHTS HAVE BEEN VIOLATED, PLEASE EMAIL US, AND ALL THE
NECESSARY CHANGES WILL BE MADE.



\renewcommand{\indexname}{Index}
\printindex
\end{document}